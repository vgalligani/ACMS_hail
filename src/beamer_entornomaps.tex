\documentclass{beamer}
\usepackage{graphicx}
\usepackage{tikz,etoolbox}
\usepackage{underscore}

%-------------- DOMAIN02
%------------------------------------------------------------------------------
% Define time steps and experiments
\def\timesEXP{18:00, 18:30, 19:00, 19:30, 20:00, 20:30, 21:00, 21:30}
\def\experiments{"WSM6_domain3_NoahMP", "P3_3MOM_LF_domain3_NoahMP","initcond_fromwrf_domain3_WSM6_d01P3_54_test2"}

\begin{document}


\newcommand{\filepath}[1]{/home/galliganiv/Work/HAILCASE_10112018/Plots/Comparison/#1}

% title slide
\begin{frame}
    \title{WRF maps (10-11-2018)}
    \maketitle
    \textbf{Experiments: WSM6(ERA5), P3(ERA5), P3(WSM6 initialization)}
\end{frame}

% loop thorugh time steps for each variable: 
\foreach \t[count=\tcount from 0] in \timesEXP {%
    \begin{frame}
    	\frametitle{1km Storm relative Helicity Comparison. Time = \t}
        \begin{figure}[h]
           \centering
           \includegraphics[width=0.99\textwidth]{\filepath{1km_Storm_Relative_Helicity_Comparison_\t.png}}
    	\end{figure}
    \end{frame}
}

% loop thorugh time steps for each variable: 
\foreach \t[count=\tcount from 0] in \timesEXP {%
    \begin{frame}
    	\frametitle{1km Storm relative Helicity Comparison. Time = \t}
        \begin{figure}[h]
           \centering
	   \includegraphics[width=0.99\textwidth]{\filepath{3km_Storm_Relative_Helicity_Comparison_\t.png}}
    	\end{figure}
    \end{frame}
}

% loop thorugh time steps for each variable:
\foreach \t[count=\tcount from 0] in \timesEXP {%
    \begin{frame}
        \frametitle{T2m. Time = \t}
        \begin{figure}[h]
           \centering
           \includegraphics[width=0.99\textwidth]{\filepath{t2m_Comparison_\t.png}}
        \end{figure}
    \end{frame}
}

% loop thorugh time steps for each variable:
\foreach \t[count=\tcount from 0] in \timesEXP {%
    \begin{frame}
        \frametitle{Td2m. Time = \t}
        \begin{figure}[h]
           \centering
           \includegraphics[width=0.99\textwidth]{\filepath{td2m_Comparison_\t.png}}
        \end{figure}
    \end{frame}
}

% loop thorugh time steps for each variable:
\foreach \t[count=\tcount from 0] in \timesEXP {%
    \begin{frame}
        \frametitle{Updraft Helicity. Time = \t}
        \begin{figure}[h]
           \centering
           \includegraphics[width=0.99\textwidth]{\filepath{Updraft_Helicity_Comparison_\t.png}}
        \end{figure}
    \end{frame}
}

% loop thorugh time steps for each variable:
\foreach \t[count=\tcount from 0] in \timesEXP {%
    \begin{frame}
        \frametitle{Winds. Time = \t}
        \begin{figure}[h]
           \centering
           \includegraphics[width=0.99\textwidth]{\filepath{Winds_Comparison_\t.png}}
        \end{figure}
    \end{frame}
}




% loop thorugh time steps for each variable: 
\foreach \t[count=\tcount from 0] in \timesEXP {%
    \begin{frame}
        \frametitle{0-6km CAPE. Time = \t}
        \begin{figure}[h]
           \centering
           \includegraphics[width=0.99\textwidth]{\filepath{mlCAPE_Comparison_\t.png}}
        \end{figure}
    \end{frame}
}


% loop thorugh time steps for each variable: 
\foreach \t[count=\tcount from 0] in \timesEXP {%
    \begin{frame}
        \frametitle{Equiv. Potential Temp. Time = \t}
        \begin{figure}[h]
           \centering
           \includegraphics[width=0.99\textwidth]{\filepath{EquivPotentialTemp_Comparison_\t.png}}
        \end{figure}
    \end{frame}
}


% loop thorugh time steps for each variable: 
\foreach \t[count=\tcount from 0] in \timesEXP {%
    \begin{frame}
        \frametitle{Convergence. Time = \t}
        \begin{figure}[h]
           \centering
           \includegraphics[width=0.99\textwidth]{\filepath{Convergence_Comparison_\t.png}}
        \end{figure}
    \end{frame}
}


% loop thorugh time steps for each variable: 
\foreach \t[count=\tcount from 0] in \timesEXP {%
    \begin{frame}
        \frametitle{1km Radar Reflectivity . Time = \t}
        \begin{figure}[h]
           \centering
	   \includegraphics[width=0.99\textwidth]{\filepath{1kRadarReflectivity_Comparison_\t.png}}
        \end{figure}
    \end{frame}
}



\end{document}

